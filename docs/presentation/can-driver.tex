\documentclass{beamer}

\usepackage[utf8]{inputenc}
\usepackage[T1]{fontenc}
\usepackage{url}

\usepackage{tikz}
\usetikzlibrary{positioning}
\usetikzlibrary{arrows}

\usepackage{listings}
\lstloadlanguages{C++}
\lstset{
	language=C++,
	basicstyle=\footnotesize,
    keywordstyle=\color{black}\bfseries,
    identifierstyle=,
    commentstyle=\color{black!75},
	tabsize=4,
	frame=tlrb,
	breaklines=true,
	captionpos=b
}

\usetheme{Luebeck}
\useoutertheme{miniframes}
\setbeamertemplate{navigation symbols}{}

\title[AVR CAN driver]{Canary - an AVR CAN driver}
\author{Christoph Steup}
\institute{EOS - IVS at FIN - OvGU}
\subject{Adaptive object oriented CAN driver for at90canX}
\keywords{avr-halib, AVR, CAN, driver, C++}


\begin{document}

\frame{\titlepage}
\frame{\frametitle{Sections}\tableofcontents}

\section{Motivation}
\begin{frame}
	\frametitle{Motivation}
	\begin{itemize}
		\item AVR is family of micro controllers, therefore:
		\begin{itemize}
			\item small amount of RAM available
			\item small amount of program memory available
			\item computational power is limited
			\item needs to fullfill realtime requirements in certain enviroments
		\end{itemize}
		\item to cope with these requirements one need:
			\begin{itemize}
				\item configurable architecture
				\item highly efficient code
				\item as much decisions at compile time as possible
			\end{itemize}
	\end{itemize}
\end{frame}

\section{Introduction}
\begin{frame}
	\frametitle{Introduction}
	\begin{itemize}
		\item Canary - CAN driver provides the fallowing features:
		\begin{itemize}
			\item adaptive configuration based on C++ template metalanguage
			\item small memory footprint
			\item small code size
			\item full usage of hardware acceleration features of the at90canX\cite{at90canX} series of micro controllers
		\end{itemize}
		\item Alternative driver CanNoInt, provides:
		\begin{itemize}
			\item highly deterministic behavior by not using interrupts
			\item even smaller footprint than Canary
			\item also smaller code size than Canary
		\end{itemize}
	\end{itemize}
\end{frame}

\section{Design}
\begin{frame}
	\frametitle{Design and Components}
	\begin{figure}[ht]
		\tikzset
		{
			compile-time/.style=
			{
				rectangle,
				minimum size=6mm,
				draw=black,
				top color=blue!50!,
				bottom color=white,
				font=\itshape
			}
		}
		\tikzset
		{
			run-time/.style=
			{
				rectangle,
				minimum size=6mm,
				draw=black,
				top color=white,
				font=\itshape
			}
		}
		\tikzset
		{
			run-time-opt/.style=
			{
				run-time,
				dashed
			}
		}
		\tikzset
		{
			interface/.style=
			{
				circle,
				inner sep=0,
				minimum size=3pt,
			}
		}
		\tikzset
		{
			if-name/.style=
			{
				rectangle,
				draw=white,
				text width=2.4cm,
				font=\itshape
			}
		}
		\begin{tikzpicture}
			\node [run-time] 		(api) 								{template free API};
			\node [compile-time] 	(config)	[left=of api] 			{Configuration};
			\node [run-time-opt] 	(ints)  	[below=of api] 			{Interrupt handlers};
			\node [run-time] 		(hal)   	[below=of ints]			{Hardware abstraction};
			\node [interface]		(api-if)	[above=of api]			{};
			\node [if-name, right]	at (api-if.south east)				{API};
			\node [interface]		(config-if)	[left=of ints]			{};
			\node [if-name, left]	at (config-if.west)					{template based configuration};
			\node [interface]		(ints-if)	[above right=of ints]	{};
			\node [interface]		(ints-coord)[right=of ints]			{};
			\node [if-name, right]	at (ints-if.east)					{Interrupt handler based interface};
			\node [interface]		(hal-if)	[above right=of hal]	{};
			\node [interface]		(hal-coord)	[right=of hal]			{};
			\node [if-name, right]	at (hal-if.east)					{Hardware independant interface};

			\path[-] (hal)	 					edge			(hal-coord.center)
					 (ints) 					edge			(ints-coord.center);
			\path[-o]
					  (hal-coord.center)		edge			(hal-if.north east)
					  (ints-coord.center) 		edge			(ints-if.north east)
					  (api)						edge			(api-if)
					  (config)					edge [blue]		(config-if.south);
			\path[)-]
					  (hal-if.south west) 			edge 			(ints.south east)
					  (ints-if.west)			edge 			(api.east)
					  (config-if.south east)	edge [blue]		(hal.north west)
					  (config-if.east)			edge [blue]		(ints.west)
					  (config-if.north east)	edge [blue]		(api.south west);
		\end{tikzpicture}
		\caption{Diagram of the components of the driver and the interfaces between them.}
	\end{figure}
\end{frame}

\section{Implementation}
\begin{frame}
	\frametitle{Basic concepts}
	\begin{itemize}
		\item Extension of the avr-halib\cite{halib}, therefore using similar concepts:
		\begin{itemize}
			\item register maps
			\item delegate style interrupt handling
		\end{itemize}
		\item some concepts, are taken from FAMOUSO\cite{famouso}:
			\begin{itemize}
			\item compile time configuration via templates
			\item using template specialization to achive optimal code
			\end{itemize}
		\item new concept: configurable regmaps, to cope with different register content in CAN 2.0A and CAN 2.0B
	\end{itemize}
\end{frame}

\begin{frame}[fragile]
	\frametitle{Example configuration}
	\begin{lstlisting}[caption=An example configuration of the Canary driver class]
	using namespace avr_halib::canary;

	struct CANConfig : defaultCANConfig{
	    typedef BaudRateConfig<F_CPU, SPEED_1M, SUBBITS_16> baudRate;

	    enum Parameters{
	        version           = CAN_20B,
	        maxConcurrentMsgs = 4,
	        useError          = false,
	        useReceive        = false
	    };
	};

	typedef Canary<CANConfig> Can;
	typedef Can::MsgSend      CanSendMsg;
	\end{lstlisting}
\end{frame}

\begin{frame}
	\frametitle{Demo network}
	\begin{itemize}
		\item small scale CAN Network - 3 Nodes:
		\begin{itemize}
			\item Sender: sends 1 message, 2 RTRs
			\item Receiver: Receives message
			\item RTR-Receiver: Respond to RTRs
			\end{itemize}
		\item laptop with pcan usb dongle to view CAN traffic
		\item whole demo interrupt driven
		\item combining CAN 2.0A and CAN 2.0B
		\item using receive and reply, as well as auto reply
	\end{itemize}
\end{frame}

\section{Evaluation}
\begin{frame}
	\frametitle{Evaluation and measurements}
	\begin{table}
	\begin{tabular}[ht]{|l|cc|}
		\hline
		application								& program size 	& ram usage \\
		\hline
		Sending with interrupts					& 1842			& 42 		\\
		Sending without interrupts				& 1528			& 42 		\\
		Receiving with interrupts, no output	& 2130			& 41 		\\
		Receiving with interrupts, LCD output	& 3602			& 42 		\\
		Receiving without interrupts, no output	& 1614			& 29 		\\
		Receiving without interrupts, LCD output& 3084			& 30 		\\
		\hline
	\end{tabular}
	\caption{Program sizes and used amount of RAM for different example applications}
	\end{table}
\end{frame}

\section{References}
\begin{frame}
	\frametitle{References}
	\bibliographystyle{plain}
	\bibliography{can-driver}
\end{frame}

\end{document}

