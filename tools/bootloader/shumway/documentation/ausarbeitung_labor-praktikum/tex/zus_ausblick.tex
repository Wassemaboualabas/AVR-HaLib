%was wurde gemacht. was wurde nicht gemacht und warum nicht. was k�nnte/sollte 
%davon noch gemacht werden
% funktioniert. verwendung avrdude passt sich in bisherige arbeitsweise ein
% nur flash, weil kein platz mehr
% langsam
% zuk�nftig? eeprom, daf�r aber platz schaffen notwendig. fuses, ebenso
\section{Zusammenfassung und Ausblick}
\label{l:zus_ausbl}
Mit der entwickelten L�sung ist es m�glich, gleichzeitig mehrere 
Mikrocontroller der AT90CAN-Baureihe �ber deren CAN-Schnittstelle zu 
programmieren. Dem Benutzer wird mit der Erweiterung des Programms 
{\it Avrdude}\/ ein entsprechendes Werkzeug zur Anwendung gegeben. 
Die gesamte Entwicklung erfolgte unter Verwendung von offen gelegter Software
unter dem Betriebssystem Linux. Da Avrdude den Quasi-Standard f�r 
Programmiersoftware unter Linux darstellt, wird der  
Aufwand zur Anwendung als minimal angenommen.
\newline
Nicht realisiert wurden M�glichkeiten zum Programmieren
der Fuse-Bytes. Ebenso fehlt ein eigener Update-Mechanismus. 
Die Entwicklung 
der genannten Punkte scheiterte am verf�gbaren Speicherplatz. Allerdings kann 
aufgrund der Architektur der L�sung angenommen werden, dass entsprechende 
Modifikationen einfach durchf�hrbar sind. Ob eine Variante 
jemals alle Bereiche programmieren kann, oder aber verschiedene spezialisierte 
Varianten existieren, wird nicht zuletzt durch den Bedarf entschieden.
Gleiches trifft auch f�r Verbesserungen hinsichtlich der 
Performance zu.

