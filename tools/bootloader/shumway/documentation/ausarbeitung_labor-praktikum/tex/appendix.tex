\appendix
\phantomsection
\addcontentsline{toc}{part}{\appendixname}

\section{Befehlprotokoll}
\label{anhang_protokoll}
%Anhang mit Progs
\pagestyle{appendix_deep}


\subsection{�ffnen}
        \begin{table}[htbp]
        \centering
        \begin{tabularx}{0.9\linewidth}{|c|c|Y|}
        \hline
        {\bf Aktion} & {\bf Daten} & {\bf Beschreibung} \\ \hline
        SELECT\_OPEN & - & �ffnet Empf�nger f�r weitere Nachrichten von 
        diesem Sender\\
        \hline
        \end{tabularx}
        \caption{\label{tab:so_req}Anfrage des Hosts nach �ffnen}
        \end{table}
        \begin{table}[htbp]
        \centering
        \begin{tabularx}{0.9\linewidth}{|c|c|Y|}
        \hline
        {\bf Aktion} & {\bf Daten} & {\bf Beschreibung} \\ \hline
        SELECT\_OPEN & OK & Empf�nger f�r weitere Nachrichten von diesem
        Sender bereit\\
        \hline
        SELECT\_OPEN & ERROR & Empf�nger konnte ungest�rten weiteren 
        Empfang nicht sicherstellen\\
        \hline
        \end{tabularx}
        \caption{\label{tab:so_resp}Antwort des Bootloaders auf �ffnen}
        \end{table}

\subsection{Schlie�en}
        \begin{table}[htbp]
        \centering
        \begin{tabularx}{0.9\linewidth}{|c|c|Y|}
        \hline
        {\bf Aktion} & {\bf Daten} & {\bf Beschreibung} \\ \hline
        SELECT\_CLOSE & - & Beenden des Empfangs weiterer Nachrichten 
        dieses Senders\\
        \hline
        \end{tabularx}
        \caption{\label{tab:sc_req}Anfrage des Hosts nach Schlie�en}
        \end{table}
        \begin{table}[htbp]
        \centering
        \begin{tabularx}{0.9\linewidth}{|c|c|Y|}
        \hline
        {\bf Aktion} & {\bf Daten} & {\bf Beschreibung} \\ \hline
        SELECT\_CLOSE & OK & Empf�nger geschlossen\\
        \hline
        \end{tabularx}
        \caption{\label{tab:sc_resp}Antwort des Bootloaders auf Schlie�en}
        \end{table}


\subsection{Speicherwahl}
        \begin{table}[htbp]
        \centering
        \begin{tabularx}{0.9\linewidth}{|c|c|Y|}
        \hline
        {\bf Aktion} & {\bf Daten} & {\bf Beschreibung} \\ \hline
        MEM\_SELECT & MEM & Auswahl einer Speichers f�r weitere Aktionen\\
        \hline
        \end{tabularx}
        \caption{\label{tab:ms_req}Anfrage des Hosts nach Wahl eines Speichers}
        \end{table}
        \begin{table}[htbp]
        \centering
        \begin{tabularx}{0.9\linewidth}{|c|c|Y|}
        \hline
        {\bf Aktion} & {\bf Daten} & {\bf Beschreibung} \\ \hline
        MEM\_SELECT & OK & Speicherwahl erfolgreich\\
        \hline
        MEM\_SELECT & ERROR & Speicherwahl ung�ltig\\
        \hline
        \end{tabularx}
        \caption{\label{tab:ms_resp}Antwort des Bootloaders auf Wahl des Speichers}
        \end{table}

\pagebreak

\subsection{Adresswahl}
        \begin{table}[h]
        \centering
        \begin{tabularx}{0.9\linewidth}{|c|c|Y|}
        \hline
        {\bf Aktion} & {\bf Daten} & {\bf Beschreibung} \\ \hline
        ADDR & Adresse & Auswahl einer Adresse f�r weitere Aktionen\\
        \hline
        \end{tabularx}
        \caption{\label{tab:as_req}Anfrage des Hosts nach Wahl der Adresse}
        \end{table}
        \begin{table}[h]
        \centering
        \begin{tabularx}{0.9\linewidth}{|c|c|Y|}
        \hline
        {\bf Aktion} & {\bf Daten} & {\bf Beschreibung} \\ \hline
        ADDR & OK & Adresswahl erfolgreich\\
        \hline
        ADDR & ERROR & Adresswahl ung�ltig\\
        \hline
        \end{tabularx}
        \caption{\label{tab:as_resp}Antwort des Bootloaders auf Wahl der Adresse}
        \end{table}

\subsection{Speicher komplett l�schen}
        \begin{table}[h]
        \centering
        \begin{tabularx}{0.9\linewidth}{|c|c|Y|}
        \hline
        {\bf Aktion} & {\bf Daten} & {\bf Beschreibung} \\ \hline
        FULL\_ERASE & - & L�schen des gesamten Speicherbereichs\\
        \hline
        \end{tabularx}
        \caption{\label{tab:fe_req}Anfrage des Hosts nach komplettem L�schen}
        \end{table}
        \begin{table}[h]
        \centering
        \begin{tabularx}{0.9\linewidth}{|c|c|Y|}
        \hline
        {\bf Aktion} & {\bf Daten} & {\bf Beschreibung} \\ \hline
        FULL\_ERASE & OK & L�schen in Durchf�hrung\\
        \hline
        FULL\_ERASE & ERROR & L�schen kann nicht durchgef�hrt werden\\
        \hline
        \end{tabularx}
        \caption{\label{tab:fe_resp}Antwort des Bootloaders auf komplettes L�schen}
        \end{table}

\subsection{Schreiben}
        \begin{table}[h]
        \centering
        \begin{tabularx}{0.9\linewidth}{|c|c|Y|}
        \hline
        {\bf Aktion} & {\bf Daten} & {\bf Beschreibung} \\ \hline
        WRITE & Programmdaten & Schreiben von Daten\\
        \hline
        \end{tabularx}
        \caption{\label{tab:w_req}Anfrage des Hosts nach Schreiben}
        \end{table}
        \begin{table}[h]
        \centering
        \begin{tabularx}{0.9\linewidth}{|c|c|Y|}
        \hline
        {\bf Aktion} & {\bf Daten} & {\bf Beschreibung} \\ \hline
        WRITE & OK & Daten akzeptiert\\
        \hline
        WRITE & ERROR & Fehler, Daten nicht akzeptiert\\
        \hline
        \end{tabularx}
        \caption{\label{tab:w_resp}Antwort des Bootloaders auf Schreiben}
        \end{table}

\pagebreak

\subsection{Lesen}
        \begin{table}[h]
        \centering
        \begin{tabularx}{0.9\linewidth}{|c|c|Y|}
        \hline
        {\bf Aktion} & {\bf Daten} & {\bf Beschreibung} \\ \hline
        READ & letzte Adresse & Lesen von Daten\\
        \hline
        \end{tabularx}
        \caption{\label{tab:r_req}Anfrage des Hosts nach Lesen}
        \end{table}
        \begin{table}[h]
        \centering
        \begin{tabularx}{0.9\linewidth}{|c|c|Y|}
        \hline
        {\bf Aktion} & {\bf Daten} & {\bf Beschreibung} \\ \hline
        READ & gelesene Daten & �bermittlung gelesener Daten\\
        \hline
        WRITE & OK & Alle Daten bis zur letzten Adresse gelesen\\
        \hline
        READ & ERROR & Fehler beim Lesen der Daten\\
        \hline
        \end{tabularx}
        \caption{\label{tab:r_resp}Antwort des Bootloaders auf Lesen}
        \end{table}

\cleardoublepage
\section{Programmiervorgang mit Avrdude}
\label{anhang_avrdude}
F�r den Programmiervorgang verwendet wurden zwei AT90CAN128. Beide befanden
sich bei Beginn im Bootloader-Modus. 
Abweichend von der �blichen Vorgehensweise, wurde 
{\it Avrdude}\/ �ber die angegebene Datei \glqq avrdude.conf\grqq\/ 
konfiguriert
\newline
\begin{verbatim}
[diederic@eoslab-09 avrdude-5.1-can]$ ./avrdude -C avrdude.conf -p at90can128 -c 
pcan -U flash:w:../../lab_prak/scratch/CAN/can.hex

avrdude: Connected to 2 nodes via CAN.
avrdude: AVR device initialized and ready to accept instructions
      
Reading | ################################################## | 100% 0.00s
        
avrdude: Device signature = 0x9d9d9d
avrdude: NOTE: FLASH memory has been specified, an erase cycle will be performed         
         To disable this feature, specify the -D option.
avrdude: current erase-rewrite cycle count is -1886417009 (if being tracked)
avrdude: erasing chip
avrdude: reading input file "../../lab_prak/scratch/CAN/can.hex"
avrdude: input file ../../lab_prak/scratch/CAN/can.hex auto detected as Intel Hex
avrdude: writing flash (2090 bytes):
        
Writing | ################################################## | 100% 1.50s
        
        
        
avrdude: 2090 bytes of flash written
avrdude: verifying flash memory against ../../lab_prak/scratch/CAN/can.hex:
avrdude: load data flash data from input file ../../lab_prak/scratch/CAN/can.hex:
avrdude: input file ../../lab_prak/scratch/CAN/can.hex auto detected as Intel Hex
avrdude: input file ../../lab_prak/scratch/CAN/can.hex contains 2090 bytes
avrdude: reading on-chip flash data:
        
Reading | ################################################## | 100% 1.63s
        
        
        
avrdude: verifying ...
avrdude: 2090 bytes of flash verified
       
avrdude: safemode: Fuses OK
        
avrdude done.  Thank you.

[diederic@eoslab-09 avrdude-5.1-can]$
\end{verbatim}
Aufgrund fehlender Unterst�tzung seitens der Entwicklung 
gibt {\it Avrdude}\/ zwei fehlerhafte Informationen aus. 
\begin{itemize}
  \item 
    Es wird keine 
Ger�tesignatur ausgelesen. Die angezeigte Signatur ist lediglich der 
zuf�llige Inhalt einer Speicherzelle. Das Auslesen einer Signatur ist von
der Entwicklung nicht vorgesehen. Dementsprechend ist eine Routine zum 
Auslesen einer Signatur nicht in {\it Avrdude}\/ implementiert. 
Eine Implementation muss aber nicht erfolgen, die Routine ist
als optional gekennzeichnet. 
\item
Es wird kein Z�hlen der Schreib- und L�schzyklen durchgef�hrt. 
{\it Avrdude}\/ erm�glicht es, vier Bytes im EEPROM abzulegen, der dann als 
Z�hler der get�tigten Zyklen dient. Das Auslesen dieser Information basiert 
wiederum auf einer als optional gekennzeichneten Routine. Obwohl diese nicht 
implementiert ist, verwendet {\it Avrdude}\/ einen vermeintlich gelesenen 
Wert und gibt diesen aus.
\end{itemize}
